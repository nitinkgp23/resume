%%%%%%%%%%%%%%%%%%%%%%%%%%%%%%%%%%%%%%%%%
% Plasmati Graduate CV
% LaTeX Template
% Version 1.0 (24/3/13)
%
% This template has been downloaded from:
% http://www.LaTeXTemplates.com
%
% Original author:
% Alessandro Plasmati (alessandro.plasmati@gmail.com)
%
% License:
% CC BY-NC-SA 3.0 (http://creativecommons.org/licenses/by-nc-sa/3.0/)
%
% Important note:
% This template needs to be compiled with XeLaTeX.
% The main document font is called Fontin and can be downloaded for free
% from here: http://www.exljbris.com/fontin.html
%
%%%%%%%%%%%%%%%%%%%%%%%%%%%%%%%%%%%%%%%%%

%----------------------------------------------------------------------------------------
%	PACKAGES AND OTHER DOCUMENT CONFIGURATIONS
%----------------------------------------------------------------------------------------

\documentclass[a4paper,10pt]{extarticle} % Default font size and paper size

\usepackage{fontspec} % For loading fonts
\defaultfontfeatures{Mapping=tex-text}
\setmainfont[SmallCapsFont = Fontin SmallCaps]{Fontin} % Main document font
\fontspec{[FontAwesome.otf]}


\usepackage{xunicode,xltxtra,url,parskip} % Formatting packages

\usepackage[usenames,dvipsnames]{xcolor} % Required for specifying custom colors

%\usepackage[big]{layaureo} % Margin formatting of the A4 page, an alternative to layaureo can be 
%\usepackage{fullpage}
\usepackage{geometry}
\geometry{a4paper,margin=0.40cm}
%\geometry{a4paper,left=20mm, top=20mm}
 %To reduce the height of the top margin uncomment: \addtolength{\voffset}{-1.3cm}

\usepackage{hyperref} % Required for adding links	and customizing them
%\definecolor{linkcolour}{rgb}{0,0.2,0.6} % Link color
\definecolor{linkcolour}{rgb}{0.3,0.3,0.3} % Link color
\hypersetup{colorlinks,breaklinks,urlcolor=linkcolour,linkcolor=linkcolour} % Set link colors throughout the document

\usepackage{titlesec} % Used to customize the \section command
\titleformat{\section}{\large\scshape\raggedright}{}{0em}{}[\titlerule] % Text formatting of sections
\titlespacing{\section}{0pt}{0pt}{0pt} % Spacing around sections

\usepackage{multicol}
\setlength{\columnsep}{0cm}

\usepackage{textcomp}

\usepackage{fontawesome}

\def\arraystretch{0.79}

\begin{document}

\pagestyle{empty} % Removes page numbering

\font\fb=''[cmr10]'' % Change the font of the \LaTeX command under the skills section

%----------------------------------------------------------------------------------------
%	NAME AND CONTACT INFORMATION
%----------------------------------------------------------------------------------------
\begin{multicols}{3}
% \par{\centering\normalsize {\textsc{Undergraduate student at Indian Institute of Technology, Kharagpur}}\par}\normalsize
% \par{\centering\normalsize {\textsc{Department of Computer Science and Engineering}}\par}\normalsize
%\par{{\begin{center}Dual Degree, \emph{Computer Science and Engineering}\end{center}}}
%\normalsize  \faGlobe\ {\href{http://nitinchoudhary.in/}{nitinchoudhary.in}}\\
\normalsize \faGithub\ {\href{https://github.com/nitinkgp23}{nitinkgp23}}\\
\normalsize  \faLinkedinSquare\ {\href{https://www.linkedin.com/in/nitin-choudhary-kgp}{nitin-choudhary-kgp}}\\
\columnbreak
\normalsize\par{\centering{\huge Nitin \textsc{Choudhary}}\par} % Your name
\par{\centering\normalsize {\textsc{E-205, Azad Hall of Residence, \\ IIT Kharagpur, West Bengal\\ India - 721302}}\hfill\par}
\columnbreak
\raggedright\hfill\normalsize \faEnvelope\ {\href{mailto:nitin.iitkgp23@gmail.com}{nitin.iitkgp23@gmail.com}}\\
\raggedright\hfill{\faPhone\ +91-8768884446}
\end{multicols}

%----------------------------------------------------------------------------------------
%	EDUCATION
%----------------------------------------------------------------------------------------

\vspace{-0.6cm}
\section{Education}

\begin{tabular}{r|p{18cm}}	
2015-2020 & Int. MSc in \textbf{Mathematics and Computing} \hfill{\textbf{CGPA : 8.33}}\\
&\textbf{Indian Institute of Technology}, Kharagpur\\
2015 & Intermediate in CBSE, \textbf{Central Academy}, Kota  \hfill{\textbf{96.2 \%}}\\
2013 & Matriculation in ICSE, \textbf{Saint Francis School}, Deoghar  \hfill{\textbf{97.2 \%}}
\end{tabular}

%----------------------------------------------------------------------------------------
%	SKILLS 
%----------------------------------------------------------------------------------------

\section{Technical Skills}

\begin{tabular}{r|p{18cm}}
\textsc{Programming} & \itshape{Proficient in} Python, C, C++ and Java\\ & \itshape{Competent in} Javascript, Lua, Matlab, Android and Shell Scripting \\
\textsc{Libraries / Frameworks} & \itshape{ML/NN: }Scikit-learn, Tensorflow, Torch, OpenCV\\
& \itshape{Others: }Numpy, Scipy, Pandas, Matplotlib, Django, Flask\\
% \textsc{Databases} & MySQL, MongoDB, PostgreSQL\\
\textsc{Systems / Platforms} & Git, Linux\\
\textsc{Markup / Templating} & HTML, CSS, LaTex
\end{tabular}

%----------------------------------------------------------------------------------------
%	Academic Projects
%----------------------------------------------------------------------------------------

\section{Academic Projects}

\begin{tabular}{r|p{18cm}}

\textsc{Feb - Apr} & \textbf{GPA Predictor using Machine Learning models and neural networks}\hfill\textbf{\href{http://www.facweb.iitkgp.ernet.in/~skbarai/}{Guide: Prof. S. K. Barai}}\\
\textsc{2017}\\
& \footnotesize{- Created an institute-level GPA predictor for a student, which would take his previous GPA's as input, and predict his GPA's in the upcoming semesters}\\
& \footnotesize{- Used last 10 years of grades for over 50 students in each department as training data, so as to identify the difficulty level of each semester.}\\
& \footnotesize{- Used	k- Nearest Neighbour alongwith SVM to increase the acceptability of the	prediction of outliers.}\\
\multicolumn{2}{c}{} \\
%--------------------------------------------
%--------------------------------------------
\textsc{Aug 2017} & \textbf{Utilising Social Media for Disaster relief managment}\hfill\textbf{\href{http://cse.iitkgp.ac.in/~sghosh/}{Guide: Prof. Saptarshi Ghosh}}\\
\textbf{Ongoing}\\
& \footnotesize{- Treating people as social sensors and utlizing their social intelligence at a disaster site, by extracting the tweets and facebook posts made, in relation to a particular disaster.}\\
& \footnotesize{- Create a post disaster management system, that would show the need and avalaibility tweets on a map based interface, so as to easily connect NGOs, volunteers and the victims to appropriate places, in real time.}\\
& \footnotesize{- Use Information Retrieval algorithms to extract only the related tweets and then apply a deep learning model to classify between the 'need' tweets and the 'availability' tweets.}\\
\multicolumn{2}{c}{} \\
%--------------------------------------------
%--------------------------------------------
\textsc{Aug 2017} & \textbf{Sanskrit text segmentation using NLP and neural networks}\hfill\textbf{\href{http://cse.iitkgp.ac.in/~pawang/}{Guide: Prof. Pawan Goyal}}\\
\textbf{Ongoing}\\
& \footnotesize{- Currently using seq2seq model approach for word segmentation and machine translation.}\\
& \footnotesize{- Experimenting with LSTM, and more complex NLP algorithms and deep learning approach to achieve the task.}\\
\multicolumn{2}{c}{} \\
%--------------------------------------------
%--------------------------------------------
%\textsc{Apr 2016} & \textbf{Term Paper on Fuzzy Logic Congestion Control in %TCP/IP in Diff-Serv Networks} \\
%& \footnotesize{}\\
%\multicolumn{2}{c}{} \\
%--------------------------------------------
%--------------------------------------------
%\textsc{Apr 2016} & \textbf{Term Paper on Automatic Detection of Landforms on %Mars using Neural Networks} \\
%& \footnotesize{}\\
%\multicolumn{2}{c}{} \\

\end{tabular}


%----------------------------------------------------------------------------------------
%	EXPERIENCE 
%----------------------------------------------------------------------------------------


\section{Experience}

\begin{tabular}{r|p{18cm}}

%--------------------------------------------
%--------------------------------------------
\textsc{May - Aug} & \textbf{Developer at Google Summer of Code} \textsc\hfill\textbf{\href{http://sunpy.org/}{SunPy under OpenAstronomy}}\\
\textsc{2017}\\
& \footnotesize{- Wrote a full-fledged high-level JSOC Client, using drms package as its backend, to download astronomical data from JSOC servers.}\\
& \footnotesize{- Wrote a full test-suite to cover the drms package, using pytest and different mock testing packages. }
\multicolumn{2}{c}{} \\
%--------------------------------------------
%--------------------------------------------
\textsc{May 2017} & \textbf{Deep Learning Intern}\hfill\textbf{\href{http://www.dewinterindia.com/}{Dewinter Opticals, New Delhi}}\\
\\
& \footnotesize{- Was solely responsible for building a Convolutional Neural Networks model, to identify between 5 different types of graphite flakes present in grey cast iron. }\\
& \footnotesize{- Worked on integrating automatic detection of graphite flakes in MaterialPlus and WeldCheck.}\\
& \footnotesize{- Used both Tensorflow and Torch as independent platforms to implement the neural network problem.
}\\
& \footnotesize{- Used OpenCV algorithms for image segmentation and stitching microscopic images.
}\\
\multicolumn{2}{c}{} \\

%--------------------------------------------
%--------------------------------------------
\textsc{Jan 2017} & \textbf{Software Developer Head}\hfill\textbf{\href{http://kossiitkgp.in/}{Kharagpur Open Source Society}}\\
\textbf{Ongoing}\\
& \footnotesize{- Conducted Kharagpur Winter of Code (KWoC), to promote open-source development in and around campus, which brought over 900+}\\
& \footnotesize{registrations, across more than 25 colleges.
}\\
& \footnotesize{- Worked as a full stack developer in building the website of KWoC, using Flask as backend, and Jekyll as the frontend.}\\
& \footnotesize{- Mentored over 50 students, in projects varying in Python and Android.}\\
\multicolumn{2}{c}{} \\

\end{tabular}

%----------------------------------------------------------------------------------------
%	TERM PAPERS 
%----------------------------------------------------------------------------------------


\section{Term Papers}

\begin{tabular}{r|p{18cm}}

\textsc{Feb 2017} & \textbf{Fuzzy Logic Congestion Control in TCP/IP in Diff-Serv Networks}\\
\\
& \footnotesize{- Use Fuzzy logic approach to achieve a better Quality of Service, by handling congestion in TCP/IP networks.}\\
& \footnotesize{- Fuzzy variables used to denote how the length of the packet queue affects the congestion, and the rate of increase of
the queue length. }\\
& \footnotesize{- Using linguistic approach to give the output whether the packet drop should be low or moderate or high.
}\\
\multicolumn{2}{c}{} \\
%--------------------------------------------
%--------------------------------------------
\textsc{Apr 2017} & \textbf{Automatic Detection of Landforms on Mars using Neural Networks}\\
\\
& \footnotesize{- Employs the use of Convolutional Neural Networks to discover volcanic unsettled cones and transversal aeolian ridges.}\\
& \footnotesize{- MarsNet, consisting of 5 different networks, was used to detect the landforms of different sizes. }\\
& \footnotesize{- Comparisons were made with results obtained from other ancient classifiers, like SVMs. }
\multicolumn{2}{c}{} \\

\end{tabular}

\newpage
\vspace*{0.3cm}
%----------------------------------------------------------------------------------------
%	COURSEWORK
%----------------------------------------------------------------------------------------

 \section{Coursework
 \hfill\small\textsc{(T)heory and (L)aboratory}}

 \begin{multicols}{2}
 - Programming and Data Structures (T/L) \\
 - Discrete Mathematics \\
 - Design and Analysis of Algorithms (T/L) \\
 - Probability and Statistics \\
 - Soft Computing Tools in Engineering \\
 - Object Oriented Software Design* (T/L) \\
 - Linear Algebra* \\
 - Computer Organisation and Architecture*
 \end{multicols}
 {\hfill{ * Currently Studying}}
 



%----------------------------------------------------------------------------------------
%	PERSONAL PROJECTS
%----------------------------------------------------------------------------------------

\section{Personal Projects}

\begin{tabular}{r|p{18cm}}

\textsc{Dec 2016} & \textbf{Scarner's Dice} \textsc{}\hfill\textbf{Android}
\vspace{1mm} \\
& \footnotesize{- Made a basic android 2-player game that works on random dice throwing. The code can be found \textbf{\href{https://github.com/nitinkgp23/ScarnersDice}{here}}}\\
\multicolumn{2}{c}{} \\

\textsc{Apr 2016} & \textbf{Birthday Bot}\hfill\textbf{Python}
\vspace{1mm} \\
& \footnotesize{- Built a automatic bot, that likes and comments on all your birthday wishes.}\\
& \footnotesize{- Uses selenium to automate the broswer to acheive the task. The code can be found \textbf{\href{https://github.com/nitinkgp23/bdaybot}{here}}}\\
\multicolumn{2}{c}{} \\

\textsc{Jan 2013} & \textbf{Railway Reservation Portal}\hfill\textbf{Java}
\vspace{1mm} \\
& \footnotesize{- Built a non-GUI railway reservation portal in Java, using object-oriented approach.}\\
& \footnotesize{- Mocked the facility of booking, editing, and cancellation of tickets and allotment of the seats using most of the real life algorithms used.}\\
\multicolumn{2}{c}{} \\

\end{tabular}

%----------------------------------------------------------------------------------------
%	OPEN SOURCE CONTRIBUTIONS
%----------------------------------------------------------------------------------------

\section{Open Source Contributions}

\begin{tabular}{r|p{18cm}}

\textsc{Python} & \textbf{Coala}
\vspace{1mm} \\
& \footnotesize{- coala provides a unified command-line interface for linting and fixing all your code, regardless of the programming languages you use.}\\
\multicolumn{2}{c}{} \\

\textsc{Python} & \textbf{Sunpy}
\vspace{1mm} \\
& \footnotesize{- Sunpy is a community-developed, free and open-source solar data analysis environment for Python.}\\
& \footnotesize{- Made a number of contributions in the package, fixing a number of bugs, and writing a full wrapper for JSOC Client to download astronomical data.}\\
\multicolumn{2}{c}{} \\

\textsc{Python} & \textbf{Drms}
\vspace{1mm} \\
& \footnotesize{- Drms is a python module for accessing HMI, AIA and MDI data, obtained from Solar Dynamics Observatory.}\\
& \footnotesize{- Wrote a full test-suite for the python module, using pytest and other mock testing packages.}\\
\multicolumn{2}{c}{} \\

\end{tabular}


%----------------------------------------------------------------------------------------
%	POSITIONS OF RESPONSIBILITY
%----------------------------------------------------------------------------------------

\section{Positions of Responsibility}

\begin{tabular}{r|p{18cm}}
%--------------------------------------------
%--------------------------------------------
\textsc{Current} & \textbf{Executive Head}, Kharagpur Open Source Society
\vspace{1mm} \\
& \footnotesize{- Conducted Kharagpur Winter of Code, a program to introduce people to open-source development, which brought over 900+ registrations.}\\
& \footnotesize{- Conducted Linux-install fest in the campus, to promote use of Linux as the preferred operating system.}\\
& \footnotesize{- Conducted Python Classes and Git workshop, to teach students the process of contributing to an open-source project.}\\
& \footnotesize{- Was the full stack developer of the website of Kharagpur Winter of Code, using Flask as backend, and Jekyll as frontend.}\\
\\
%--------------------------------------------
%--------------------------------------------
\textsc{Current} & \textbf{Web Secretary}, Mathematics Colloquium, IIT Kharagpur
\vspace{1mm} \\
& \footnotesize{- Managing the official website of the Department of Mathematics.}\\
& \footnotesize{- Managing the development of the student portal, which gives access to all study materials and question papers related to course subjects.}\\
\\
%--------------------------------------------
%--------------------------------------------
\textsc{Current} & \textbf{Senior Editor}, Technology Literary Society, IIT Kharagpur 
\vspace{1mm} \\
& \footnotesize{- Managing the content and design team of the society.}\\
& \footnotesize{- Writer in the English Team, and working as a senior editor for all English publications.}\\
\\
%--------------------------------------------
%--------------------------------------------
\textsc{Jul - Dec} & \textbf{Core Team Member}, Space Technology Students' Society, IIT Kharagpur
\vspace{1mm} \\
\textsc{2015}& \footnotesize{- Acted as Junior Coordinator in National Students' Space Challenge, India's largest space tech-fest.}\\
& \footnotesize{- Involved in conducting various space-related events and seminars in the campus.}\\
\end{tabular}


%----------------------------------------------------------------------------------------
%	ACHIEVEMENTS
%----------------------------------------------------------------------------------------

 \section{Scholastic Achievements}

\begin{tabular}{r|p{18cm}}
\textsc{Current} &{Recipient of Innovation of Science Pursuit for Inspire Research (INSPIRE) Scholarship}
\vspace{2mm} \\
\textsc{2011} &{Secured All India Rank 2, in National Cyber Olympiad in high school}
\vspace{2mm} \\
\textsc{2015} &{Secured 99.11 percentile in JEE Advanced 2015}
\vspace{2mm} \\
\textsc{2015} &{Secured 99.33 percentile in JEE Mains 2015}
\vspace{2mm} \\ 
\textsc{2012} &{State-level awardee at National Children Science Congress} \vspace{2mm} \\
\end{tabular}

%----------------------------------------------------------------------------------------

%\newpage
%----------------------------------------------------------------------------------------

\end{document}